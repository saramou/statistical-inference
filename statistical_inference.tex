% Options for packages loaded elsewhere
\PassOptionsToPackage{unicode}{hyperref}
\PassOptionsToPackage{hyphens}{url}
%
\documentclass[
]{article}
\usepackage{lmodern}
\usepackage{amssymb,amsmath}
\usepackage{ifxetex,ifluatex}
\ifnum 0\ifxetex 1\fi\ifluatex 1\fi=0 % if pdftex
  \usepackage[T1]{fontenc}
  \usepackage[utf8]{inputenc}
  \usepackage{textcomp} % provide euro and other symbols
\else % if luatex or xetex
  \usepackage{unicode-math}
  \defaultfontfeatures{Scale=MatchLowercase}
  \defaultfontfeatures[\rmfamily]{Ligatures=TeX,Scale=1}
\fi
% Use upquote if available, for straight quotes in verbatim environments
\IfFileExists{upquote.sty}{\usepackage{upquote}}{}
\IfFileExists{microtype.sty}{% use microtype if available
  \usepackage[]{microtype}
  \UseMicrotypeSet[protrusion]{basicmath} % disable protrusion for tt fonts
}{}
\makeatletter
\@ifundefined{KOMAClassName}{% if non-KOMA class
  \IfFileExists{parskip.sty}{%
    \usepackage{parskip}
  }{% else
    \setlength{\parindent}{0pt}
    \setlength{\parskip}{6pt plus 2pt minus 1pt}}
}{% if KOMA class
  \KOMAoptions{parskip=half}}
\makeatother
\usepackage{xcolor}
\IfFileExists{xurl.sty}{\usepackage{xurl}}{} % add URL line breaks if available
\IfFileExists{bookmark.sty}{\usepackage{bookmark}}{\usepackage{hyperref}}
\hypersetup{
  pdftitle={Statistical inference course project},
  hidelinks,
  pdfcreator={LaTeX via pandoc}}
\urlstyle{same} % disable monospaced font for URLs
\usepackage[margin=1in]{geometry}
\usepackage{color}
\usepackage{fancyvrb}
\newcommand{\VerbBar}{|}
\newcommand{\VERB}{\Verb[commandchars=\\\{\}]}
\DefineVerbatimEnvironment{Highlighting}{Verbatim}{commandchars=\\\{\}}
% Add ',fontsize=\small' for more characters per line
\usepackage{framed}
\definecolor{shadecolor}{RGB}{248,248,248}
\newenvironment{Shaded}{\begin{snugshade}}{\end{snugshade}}
\newcommand{\AlertTok}[1]{\textcolor[rgb]{0.94,0.16,0.16}{#1}}
\newcommand{\AnnotationTok}[1]{\textcolor[rgb]{0.56,0.35,0.01}{\textbf{\textit{#1}}}}
\newcommand{\AttributeTok}[1]{\textcolor[rgb]{0.77,0.63,0.00}{#1}}
\newcommand{\BaseNTok}[1]{\textcolor[rgb]{0.00,0.00,0.81}{#1}}
\newcommand{\BuiltInTok}[1]{#1}
\newcommand{\CharTok}[1]{\textcolor[rgb]{0.31,0.60,0.02}{#1}}
\newcommand{\CommentTok}[1]{\textcolor[rgb]{0.56,0.35,0.01}{\textit{#1}}}
\newcommand{\CommentVarTok}[1]{\textcolor[rgb]{0.56,0.35,0.01}{\textbf{\textit{#1}}}}
\newcommand{\ConstantTok}[1]{\textcolor[rgb]{0.00,0.00,0.00}{#1}}
\newcommand{\ControlFlowTok}[1]{\textcolor[rgb]{0.13,0.29,0.53}{\textbf{#1}}}
\newcommand{\DataTypeTok}[1]{\textcolor[rgb]{0.13,0.29,0.53}{#1}}
\newcommand{\DecValTok}[1]{\textcolor[rgb]{0.00,0.00,0.81}{#1}}
\newcommand{\DocumentationTok}[1]{\textcolor[rgb]{0.56,0.35,0.01}{\textbf{\textit{#1}}}}
\newcommand{\ErrorTok}[1]{\textcolor[rgb]{0.64,0.00,0.00}{\textbf{#1}}}
\newcommand{\ExtensionTok}[1]{#1}
\newcommand{\FloatTok}[1]{\textcolor[rgb]{0.00,0.00,0.81}{#1}}
\newcommand{\FunctionTok}[1]{\textcolor[rgb]{0.00,0.00,0.00}{#1}}
\newcommand{\ImportTok}[1]{#1}
\newcommand{\InformationTok}[1]{\textcolor[rgb]{0.56,0.35,0.01}{\textbf{\textit{#1}}}}
\newcommand{\KeywordTok}[1]{\textcolor[rgb]{0.13,0.29,0.53}{\textbf{#1}}}
\newcommand{\NormalTok}[1]{#1}
\newcommand{\OperatorTok}[1]{\textcolor[rgb]{0.81,0.36,0.00}{\textbf{#1}}}
\newcommand{\OtherTok}[1]{\textcolor[rgb]{0.56,0.35,0.01}{#1}}
\newcommand{\PreprocessorTok}[1]{\textcolor[rgb]{0.56,0.35,0.01}{\textit{#1}}}
\newcommand{\RegionMarkerTok}[1]{#1}
\newcommand{\SpecialCharTok}[1]{\textcolor[rgb]{0.00,0.00,0.00}{#1}}
\newcommand{\SpecialStringTok}[1]{\textcolor[rgb]{0.31,0.60,0.02}{#1}}
\newcommand{\StringTok}[1]{\textcolor[rgb]{0.31,0.60,0.02}{#1}}
\newcommand{\VariableTok}[1]{\textcolor[rgb]{0.00,0.00,0.00}{#1}}
\newcommand{\VerbatimStringTok}[1]{\textcolor[rgb]{0.31,0.60,0.02}{#1}}
\newcommand{\WarningTok}[1]{\textcolor[rgb]{0.56,0.35,0.01}{\textbf{\textit{#1}}}}
\usepackage{graphicx,grffile}
\makeatletter
\def\maxwidth{\ifdim\Gin@nat@width>\linewidth\linewidth\else\Gin@nat@width\fi}
\def\maxheight{\ifdim\Gin@nat@height>\textheight\textheight\else\Gin@nat@height\fi}
\makeatother
% Scale images if necessary, so that they will not overflow the page
% margins by default, and it is still possible to overwrite the defaults
% using explicit options in \includegraphics[width, height, ...]{}
\setkeys{Gin}{width=\maxwidth,height=\maxheight,keepaspectratio}
% Set default figure placement to htbp
\makeatletter
\def\fps@figure{htbp}
\makeatother
\setlength{\emergencystretch}{3em} % prevent overfull lines
\providecommand{\tightlist}{%
  \setlength{\itemsep}{0pt}\setlength{\parskip}{0pt}}
\setcounter{secnumdepth}{-\maxdimen} % remove section numbering

\title{Statistical inference course project}
\author{}
\date{\vspace{-2.5em}}

\begin{document}
\maketitle

\hypertarget{r-markdown}{%
\subsection{R Markdown}\label{r-markdown}}

Part 1: Simulation Exercise Instructions :

In this project you will investigate the exponential distribution in R
and compare it with the Central Limit Theorem. The exponential
distribution can be simulated in R with rexp(n, lambda) where lambda is
the rate parameter. Set lambda = 0.2 for all of the simulations. You
will investigate the distribution of averages of 40 exponentials. Note
that you will need to do a thousand simulations.

Simulating the exponential distribution :

\begin{Shaded}
\begin{Highlighting}[]
\KeywordTok{set.seed}\NormalTok{(}\DecValTok{16}\NormalTok{)}
\NormalTok{lambda <-}\FloatTok{0.2}
\NormalTok{n <-}\StringTok{ }\DecValTok{40}
\NormalTok{sim <-}\StringTok{ }\DecValTok{1000}
\CommentTok{#We create 1K times a sample of 40 exponentials, then compute its mean}
\NormalTok{exp_dis_sample <-}\StringTok{ }\KeywordTok{replicate}\NormalTok{ (sim, }\KeywordTok{mean}\NormalTok{(}\KeywordTok{rexp}\NormalTok{ (n,lambda)))}
\end{Highlighting}
\end{Shaded}

Calculating the sample mean :

\begin{Shaded}
\begin{Highlighting}[]
\KeywordTok{mean}\NormalTok{(exp_dis_sample)}
\end{Highlighting}
\end{Shaded}

\begin{verbatim}
## [1] 4.994828
\end{verbatim}

Calculating the theoretical mean :

\begin{Shaded}
\begin{Highlighting}[]
\DecValTok{1}\OperatorTok{/}\NormalTok{lambda}
\end{Highlighting}
\end{Shaded}

\begin{verbatim}
## [1] 5
\end{verbatim}

\begin{Shaded}
\begin{Highlighting}[]
\CommentTok{##There is a very small difference between the theoretical and sample means of this exponential dsitribution}
\KeywordTok{abs}\NormalTok{(}\KeywordTok{mean}\NormalTok{(exp_dis_sample)}\OperatorTok{-}\DecValTok{1}\OperatorTok{/}\NormalTok{lambda)}
\end{Highlighting}
\end{Shaded}

\begin{verbatim}
## [1] 0.005172086
\end{verbatim}

Calculating the sample variance :

\begin{Shaded}
\begin{Highlighting}[]
\KeywordTok{var}\NormalTok{(exp_dis_sample)}
\end{Highlighting}
\end{Shaded}

\begin{verbatim}
## [1] 0.6132857
\end{verbatim}

Calculating the theoretical variance for a population of n:

\begin{Shaded}
\begin{Highlighting}[]
\NormalTok{(}\DecValTok{1}\OperatorTok{/}\NormalTok{lambda)}\OperatorTok{^}\DecValTok{2}\OperatorTok{/}\NormalTok{n}
\end{Highlighting}
\end{Shaded}

\begin{verbatim}
## [1] 0.625
\end{verbatim}

\begin{Shaded}
\begin{Highlighting}[]
\CommentTok{##There is a very small difference between the theoretical and sample variances of this exponential dsitribution}
\KeywordTok{abs}\NormalTok{(}\KeywordTok{var}\NormalTok{(exp_dis_sample)}\OperatorTok{-}\NormalTok{(}\DecValTok{1}\OperatorTok{/}\NormalTok{lambda)}\OperatorTok{^}\DecValTok{2}\OperatorTok{/}\NormalTok{n)}
\end{Highlighting}
\end{Shaded}

\begin{verbatim}
## [1] 0.01171433
\end{verbatim}

Plotting the distribution, and comparing it to a normal distribution of
similar mean and standard deviation :
\includegraphics{statistical_inference_files/figure-latex/Histogram distribution-1.pdf}

Part 2: Basic Inferential Data Analysis Instructions :

Now in the second portion of the project, we're going to analyze the
ToothGrowth data in the R datasets package.

\begin{Shaded}
\begin{Highlighting}[]
\KeywordTok{library}\NormalTok{(datasets)}
\NormalTok{df<-}\KeywordTok{data}\NormalTok{(ToothGrowth)}
\CommentTok{#First rows of the data sets}
\KeywordTok{head}\NormalTok{(ToothGrowth)}
\end{Highlighting}
\end{Shaded}

\begin{verbatim}
##    len supp dose
## 1  4.2   VC  0.5
## 2 11.5   VC  0.5
## 3  7.3   VC  0.5
## 4  5.8   VC  0.5
## 5  6.4   VC  0.5
## 6 10.0   VC  0.5
\end{verbatim}

\begin{Shaded}
\begin{Highlighting}[]
\CommentTok{#length of the dataset}
\KeywordTok{dim}\NormalTok{(ToothGrowth)}
\end{Highlighting}
\end{Shaded}

\begin{verbatim}
## [1] 60  3
\end{verbatim}

\begin{Shaded}
\begin{Highlighting}[]
\CommentTok{#Some statistics summaries of the dataset's columns}
\KeywordTok{summary}\NormalTok{(ToothGrowth)}
\end{Highlighting}
\end{Shaded}

\begin{verbatim}
##       len        supp         dose      
##  Min.   : 4.20   OJ:30   Min.   :0.500  
##  1st Qu.:13.07   VC:30   1st Qu.:0.500  
##  Median :19.25           Median :1.000  
##  Mean   :18.81           Mean   :1.167  
##  3rd Qu.:25.27           3rd Qu.:2.000  
##  Max.   :33.90           Max.   :2.000
\end{verbatim}

\begin{Shaded}
\begin{Highlighting}[]
\CommentTok{# Display different summary}
\KeywordTok{str}\NormalTok{(ToothGrowth)}
\end{Highlighting}
\end{Shaded}

\begin{verbatim}
## 'data.frame':    60 obs. of  3 variables:
##  $ len : num  4.2 11.5 7.3 5.8 6.4 10 11.2 11.2 5.2 7 ...
##  $ supp: Factor w/ 2 levels "OJ","VC": 2 2 2 2 2 2 2 2 2 2 ...
##  $ dose: num  0.5 0.5 0.5 0.5 0.5 0.5 0.5 0.5 0.5 0.5 ...
\end{verbatim}

\begin{Shaded}
\begin{Highlighting}[]
\CommentTok{#Only 3 values for dose}
\KeywordTok{unique}\NormalTok{(ToothGrowth}\OperatorTok{$}\NormalTok{dose)}
\end{Highlighting}
\end{Shaded}

\begin{verbatim}
## [1] 0.5 1.0 2.0
\end{verbatim}

\begin{Shaded}
\begin{Highlighting}[]
\CommentTok{#Some plots to better understand the data :}
\CommentTok{#Transform dose into factors}
\NormalTok{ToothGrowth}\OperatorTok{$}\NormalTok{dose<-}\KeywordTok{as.factor}\NormalTok{(ToothGrowth}\OperatorTok{$}\NormalTok{dose)}
\CommentTok{#contruct a boxplot}
\KeywordTok{library}\NormalTok{(ggplot2)}
\end{Highlighting}
\end{Shaded}

\begin{verbatim}
## Warning: package 'ggplot2' was built under R version 4.0.3
\end{verbatim}

\begin{Shaded}
\begin{Highlighting}[]
\KeywordTok{ggplot}\NormalTok{(}\DataTypeTok{data =}\NormalTok{ ToothGrowth, }\KeywordTok{aes}\NormalTok{(}\DataTypeTok{x =}\NormalTok{ dose, }\DataTypeTok{y =}\NormalTok{ len)) }\OperatorTok{+}\StringTok{ }
\StringTok{  }\KeywordTok{geom_boxplot}\NormalTok{(}\KeywordTok{aes}\NormalTok{(}\DataTypeTok{fill =}\NormalTok{ dose), }\DataTypeTok{width =} \FloatTok{0.8}\NormalTok{) }\OperatorTok{+}\StringTok{ }\KeywordTok{xlab}\NormalTok{(}\StringTok{'Dose'}\NormalTok{) }\OperatorTok{+}\StringTok{ }\KeywordTok{ylab}\NormalTok{(}\StringTok{'Tooth length'}\NormalTok{) }\OperatorTok{+}\StringTok{ }\KeywordTok{facet_grid}\NormalTok{(}\OperatorTok{~}\NormalTok{supp) }\OperatorTok{+}\KeywordTok{ggtitle}\NormalTok{(}\StringTok{'Tooth Length vs. Doses, by supp'}\NormalTok{)}
\end{Highlighting}
\end{Shaded}

\includegraphics{statistical_inference_files/figure-latex/tootgrowth-1.pdf}

Calculating confidence intervals :

Hypothesis1 : The supplements offer the same tooth growth

\begin{Shaded}
\begin{Highlighting}[]
\CommentTok{# Run a t-test on the data}
\KeywordTok{t.test}\NormalTok{(len}\OperatorTok{~}\NormalTok{supp,}\DataTypeTok{data=}\NormalTok{ToothGrowth)}
\end{Highlighting}
\end{Shaded}

\begin{verbatim}
## 
##  Welch Two Sample t-test
## 
## data:  len by supp
## t = 1.9153, df = 55.309, p-value = 0.06063
## alternative hypothesis: true difference in means is not equal to 0
## 95 percent confidence interval:
##  -0.1710156  7.5710156
## sample estimates:
## mean in group OJ mean in group VC 
##         20.66333         16.96333
\end{verbatim}

the p-value is 6\%, so higher than 5\%, and the confidence interval
contains 0. The null hypothesis cannot be rejected.

Hypothesis2 : The supplements offer the same tooth growth For the dosage
of `0.5mg/day'

\begin{Shaded}
\begin{Highlighting}[]
\CommentTok{# Run a t-test on the data for dose .5mg/day}
\NormalTok{df<-}\KeywordTok{subset}\NormalTok{(ToothGrowth,dose}\OperatorTok{==}\NormalTok{.}\DecValTok{5}\NormalTok{)}
\KeywordTok{t.test}\NormalTok{(len}\OperatorTok{~}\NormalTok{supp,}\DataTypeTok{data=}\NormalTok{df)}
\end{Highlighting}
\end{Shaded}

\begin{verbatim}
## 
##  Welch Two Sample t-test
## 
## data:  len by supp
## t = 3.1697, df = 14.969, p-value = 0.006359
## alternative hypothesis: true difference in means is not equal to 0
## 95 percent confidence interval:
##  1.719057 8.780943
## sample estimates:
## mean in group OJ mean in group VC 
##            13.23             7.98
\end{verbatim}

the p-value is .6\%, so less than 5\%, and the confidence interval is
higher than 0. So the results are statistically significant and the null
hypothesis is rejected. The OJ supp offer more tooth growth than VC.

\begin{Shaded}
\begin{Highlighting}[]
\CommentTok{# Run a t-test on the data for dose 1mg/day}
\NormalTok{df<-}\KeywordTok{subset}\NormalTok{(ToothGrowth,dose}\OperatorTok{==}\DecValTok{1}\NormalTok{)}
\KeywordTok{t.test}\NormalTok{(len}\OperatorTok{~}\NormalTok{supp,}\DataTypeTok{data=}\NormalTok{df)}
\end{Highlighting}
\end{Shaded}

\begin{verbatim}
## 
##  Welch Two Sample t-test
## 
## data:  len by supp
## t = 4.0328, df = 15.358, p-value = 0.001038
## alternative hypothesis: true difference in means is not equal to 0
## 95 percent confidence interval:
##  2.802148 9.057852
## sample estimates:
## mean in group OJ mean in group VC 
##            22.70            16.77
\end{verbatim}

the same is true for a dosage of 1mg/day

\begin{Shaded}
\begin{Highlighting}[]
\CommentTok{# Run a t-test on the data for dose 2mg/day}
\NormalTok{df<-}\KeywordTok{subset}\NormalTok{(ToothGrowth,dose}\OperatorTok{==}\DecValTok{2}\NormalTok{)}
\KeywordTok{t.test}\NormalTok{(len}\OperatorTok{~}\NormalTok{supp,}\DataTypeTok{data=}\NormalTok{df)}
\end{Highlighting}
\end{Shaded}

\begin{verbatim}
## 
##  Welch Two Sample t-test
## 
## data:  len by supp
## t = -0.046136, df = 14.04, p-value = 0.9639
## alternative hypothesis: true difference in means is not equal to 0
## 95 percent confidence interval:
##  -3.79807  3.63807
## sample estimates:
## mean in group OJ mean in group VC 
##            26.06            26.14
\end{verbatim}

Unfortunately for a dosage of 2mg/day, the p-value is very high, and the
confidence interval contains 0. So the null hypothesis cannot be
rejected.

Conclusion : * For the dosages of .5 and 1mg/day, the OJ supplement
offers more tooth growth than VC. However, we failed to reject the null
hypothesis for the dosage 2mg/day and conclude there is no significant
difference between both supplements.

\end{document}
